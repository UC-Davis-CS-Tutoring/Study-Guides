\chapter{Set Theory}
Set Theory is one of the most fundamental topics in Abstract/Discrete
Mathematics. If you have ever heard about Relations, Equivalence Classes,
Groups, Rings, and Fields, they are all based off Set Theory! This chapter
covers the following and it is essential that you know all of it:

\section{Definitions}
\underline{\textbf{Definition:}} Set A is said to be a
\underline{\textbf{subset}} of the set B if ($\forall x \in A$)[$x \in B$]. \\
\underline{\textbf{Definition:}} The \underline{\textbf{union}} of set A and
set B is the set: \{$x \mid x \in A \vee x \in B$\}. \\
\underline{\textbf{Definition:}} The \underline{\textbf{intersection}} of A and
B is the set: \{$x \mid x \in A \wedge x \in B$\}.\\
\underline{\textbf{Definition:}} The \underline{\textbf{difference}} of A and B
is the set: \{$x \mid x \in A \wedge x \not\in B$\}. \\
\underline{\textbf{Definition:}} Sets A and B are said to be
\underline{\textbf{equal}} if $(A \subset B) \wedge (B \subset A)$. \\
\underline{\textbf{Definition:}} The \underline{\textbf{complement}} of A is
the set: \{$x \mid x \in D \wedge x \not\in A$\}, where D is the domain or
universal set. \\
\underline{\textbf{Definition:}} The \underline{\textbf{cardinality}} of A is
the number of elements in A.

\section{Proofs}
This class in particular will not test too many hard proofs in Set Theory. In
fact, most proofs in this chapter will require you to construct simple direct
proofs. So, let us show you an example of how proofs in set theory might look
like: \\
\begin{center}
    If A and B are two sets, show that $\overline{A \cap B} = \overline{A} \cup
    \overline{B}$.
\end{center}

\begin{proof}
    We need to show that $\overline{A \cap B} \subset \overline{A} \cup
    \overline{B}$ and $\overline{A} \cup \overline{B} \subset \overline{A \cap
    B}$ \\
    Let x be an arbitrary element $\in \overline{A \cap B}$
    \begin{center}
        Then, $x \not\in A \cap B$ \\
        Then, $\neg(x \in A \wedge x \in B)$ \\
        Then, $x \not\in A \vee x \not\in B$ \\
        Then, $x \in \overline{A} \cup \overline{B}$ \\
        Then, $\overline{A \cap B} \subset \overline{A} \cup \overline{B}$
    \end{center}
    Let $x \in \overline{A} \cup \overline{B}$
    \begin{center}
        Then, $x \in \overline{A} \vee x \in \overline{B}$ \\
        Then, $\neg x \in A \vee \neg x \in B$ \\
        Then, $\neg (x \in A \wedge x \in B)$ \\
        Then, $\neg (x \in A \cap B)$ \\
        Then, $x \in \overline{A \cap B}$ \\
        Thus, $\overline{A} \cup \overline{B} \subset \overline{A \cap B}$
    \end{center}
    Thus, $\overline{A \cap B} = \overline{A} \cup \overline{B}$.
\end{proof}
You also need to know set identities but they are the same as the basic
identities of propositional logic, so I haven't included them in this section.
\\

I have also skipped the section of Generalized Unions and Intersections as they
are based on the basic definition of union and intersection of sets. Otherwise,
that is pretty much it with regards to set theory!
