\documentclass[a4paper,11pt]{book}
\usepackage{amsmath}
\usepackage{amssymb}
\usepackage{amsthm}
\usepackage{booktabs}
\usepackage[T1]{fontenc}
\usepackage[utf8]{inputenc}
\usepackage{lmodern}
\usepackage{hyperref}
\usepackage{graphicx}
\usepackage{pgfplots}
\usepackage{enumitem}
\usepackage{mathtools}
\pgfplotsset{width=10cm,compat=1.9}
\usepackage[english]{babel}

% The minipage prevents LaTeX from putting linebreaks in lists
\newenvironment{symbollist} {\begin{minipage}{\linewidth}
    \begin{description}[labelindent=1em,itemsep=0em,parsep=0em]}
    {\end{description}\end{minipage}}

\newenvironment{caselist}[1][Case \arabic*:]{\begin{minipage}{\linewidth}
    \begin{enumerate}[label=\textbf{\emph{#1}},itemsep=0em,parsep=0.5em,leftmargin=*]}
    {\end{enumerate}\end{minipage}}

\newenvironment{trylist}{\begin{caselist}[Try:]}{\end{caselist}}

\newtheorem*{theorem}{Theorem}
% https://tex.stackexchange.com/questions/37472/spacing-before-and-after-with-newtheoremstyle
\newtheoremstyle{nospace}
  {0} % Space above
  {0} % Space below
  {} % Body font
  {} % Indent amount
  {\bfseries} % Theorem head font
  {.} % Punctuation after theorem head
  {.5em} % Space after theorem head
  {} % Theorem head spec (can be left empty, meaning `normal')
\theoremstyle{nospace} \newtheorem*{definition}{Definition}
\theoremstyle{definition} \newtheorem*{question}{Question}
\newtheorem*{solution}{Solution}
\newtheorem*{remark}{Remark}
\newtheorem*{example}{Example}

%%%%%%%%%%%%%%%%%%%%%%%%%%%%%%%%%%%%%%%%%%%%%%%%%%%
% First page of book which contains 'stuff' like: %
%  - Book title, subtitle                         %
%  - Book author name                             %
%%%%%%%%%%%%%%%%%%%%%%%%%%%%%%%%%%%%%%%%%%%%%%%%%%%

% Book's title and subtitle
\title{\Huge \textbf{Study Guide}  \\ \huge ECS 20: Discrete Math for CS
% Author
\author{\textsc{Computer Science Tutoring Club} \\ \textsc{Fall 2017}
}
}


\begin{document}
\frontmatter
\maketitle

%%%%%%%%%%%%%%%%%%%%%%%%%%%%%%%%%%%%%%%%%%%%%%%%%%%%%%%%%%%%%%%%%%%%%%%%
% Auto-generated table of contents, list of figures and list of tables %
%%%%%%%%%%%%%%%%%%%%%%%%%%%%%%%%%%%%%%%%%%%%%%%%%%%%%%%%%%%%%%%%%%%%%%%%
\tableofcontents

\mainmatter

%%%%%%%%%%%
% Preface %
%%%%%%%%%%%
\chapter*{How to Use This Study Guide}
\setlength{\parskip}{1em}
This study guide is meant to help you review topics before the midterms and the
final. Please do not use this study guide as your main source of preparation
for exams.

This study guide goes over each and every topic in ECS 20. I have previously
taken ECS 20 with Professor Koehl, so I have a pretty good idea about exams and
the material covered. Hence this study guide is tailor made for students in
Professor Patrice Koehl's ECS 20. If you are not in Professor Koehl's ECS 20
and still wish to use this study guide, please go ahead and do so. Please be
forewarned that some topics might not covered in this study guide that other
professors might cover (for example: Graph Theory is not covered in this study
guide).

Each chapter in the study guide reviews important concepts and theorems.
Furthermore, I have gone ahead and solved some questions that have been asked
on previous midterms. Additionally, I have added a few questions for you to try
out on your own (\textbf{No solutions have been provided}).

I sincerely hope that this study guide helps you better prepare for the
midterms and the final! Good luck on this challenging course.

\noindent \textbf{Aakash Prabhu (Class of 2019)} \\
\noindent \textbf{President, \emph{Computer Science Tutoring Club}}

\setlength{\parskip}{0em}

\chapter{Math Symbols, Notations, and Identities}

\section{Symbols and Notations}
Here are the symbols and notations that you must absolutely know!

\subsection{Sets of Numbers}
\begin{symbollist}
    \item[$\mathbb{N}$] = Set of all Natural Numbers.
    \item[$\mathbb{Z}$] = Set of all Integers.
    \item[$\mathbb{Q}$] = Set of all Rational Numbers.
    \item[$\mathbb{R}$] = Set of all Real Numbers.
    \item[$\mathbb{C}$] = Set of all Complex Numbers.
\end{symbollist}

\subsection{Propositional Logic}
\begin{symbollist}
    \item[$\neg P$] = Negation of the proposition, \emph{P}.
    \item[$P \wedge Q$] = Conjunction of the propositions \emph{P and Q}.
    \item[$P \vee Q$] = Disjunction of the propositions \emph{P and Q}.
    \item[$P \oplus Q$] = P XOR Q (Exclusive Or).
    \item[$P \rightarrow Q$] = "If P, then Q" (Conditional).
    \item[$P \leftrightarrow Q$] = "P if and only if Q" (Biconditional).
    \item[$P \Leftrightarrow Q$] = P is equivalent to Q.
\end{symbollist}

% This would prevent LaTeX from "spreading out" the text to fill up the whole
% page, which I personally prefer, but isn't necessary

% \vfill

\subsection{Set Theory}
\begin{symbollist}
    \item[$\emptyset$] = Empty Set.
    \item[$ x \in \mathbb{Q}$] = x belongs to a rational number.
    \item[$A \subset B$] = Set A is a subset of set B.
    \item[$S = \{1,2,3,4\}$] =  An example of a set in roster form.
    \item[$S = \{x \mid 1 \leq x \leq 4, x \in \mathbb{Z}\}$] = An example of a
        set in set-builder form.
    \item[$A \cup B$] = A union B.
    \item[$A \cap B$] = A intersection B.
    \item[$A - B$] = A minus B (Set difference).
    \item[$\overline{A}$] = Complement of A.
    \item[$\mid A \mid$] = Number of elements in A (Cardinality).
\end{symbollist}


\subsection{Functions}
\begin{symbollist}
    \item[$\lceil x \rceil $] = Ceiling of x.
    \item[$\lfloor x \rfloor$] = Floor of x.
    \item[$O(f(x))$] = Big-O of the function \emph{f}.
    \item[$\Omega(f(x))$] = Big-Omega of the function \emph{f}.
    \item[$\Theta(f(x))$] = Big-Theta of the function \emph{f}.
\end{symbollist}

\subsection{Number Theory}
\begin{symbollist}
    \item[$a / b$] = a divides b.
    \item[$a \equiv b \lbrack m \rbrack$] = a is congruent to b modulo m.
\end{symbollist}

\subsection{Miscellaneous}
\begin{symbollist}
    \item[$\sum_{i = 1} ^ {n} i$] = Sum of first n terms.
    \item[$\prod_{i = 1} ^ {n} i$] = Product of first n terms.  $n!$ = \emph{n}
        factorial.
    \item[$\forall x \in \mathbb{Z}, P(x)$] = For all x in the set of integers,
        P(x) is true.
    \item[$\exists x \in \mathbb{Z}, P(x)$] = There exists an x such that P(x)
        is true.
\end{symbollist}

\pagebreak[4]

\section{Important Identities}
\indent \indent I have compiled a bunch of identities that are going to prove
to be very useful for proofs or problem solving. Again, I have broken them down
into chapters for better reference \footnote{I haven't included basic
conjunctions and disjunctions of propositions because you should be knowing
them by now!}.

\subsection{Simple Mathematical Identities}
\begin{symbollist}
    \item $(a + b)^{2} = a^{2} + 2ab + b^{2}$
    \item $(a - b)^{2} = a^{2} - 2ab + b^{2}$
    \item $a^{2} - b^{2} = (a + b)\times(a - b)$
    \item $a^{m} \times a^{n} = a^{m + n}$
    \item $(a^{m})^{n} = a^{mn}$
    \item $\log(a \times b) = \log(a) + \log(b)$
    \item $\log(a^{b}) = b \times \log(a)$
\end{symbollist}

\subsection{Propositional Logic}
\begin{symbollist}
    \item $\neg (\neg P) = P$
    \item $P \oplus Q = (P \vee Q) \wedge (\neg(P \wedge Q))$
    \item $\neg(P \wedge Q) = \neg P \vee \neg Q$
    \item $\neg(P \vee Q) = \neg P \wedge \neg Q$
    \item $P \vee P = P$
    \item $P \wedge P = P$
    \item $P \wedge Q = Q \wedge P$
    \item $P \vee Q = Q \vee P$
    \item $(P \wedge Q) \wedge R = P \wedge (Q \wedge R)$
    \item $(P \vee Q) \vee R = P \vee (Q \vee R)$
    \item $P \wedge (Q \vee R) = (P \vee Q) \wedge (P \vee R)$
    \item $P \vee (Q \wedge R) = (P \wedge R) \vee (P \wedge R)$
    \item $P \leftrightarrow Q \Leftrightarrow (P \rightarrow Q) \wedge (Q
    \rightarrow P)$
    \item $P \rightarrow Q = \neg P \vee Q$
\end{symbollist}

\subsection{Functions}
\begin{symbollist}
    \item $ x - 1 < \lfloor x \rfloor \leq x \leq \lceil x \rceil < x + 1$
    \item $ \lfloor - x \rfloor = - \lceil x \rceil$
    \item $ \lceil -x \rceil = - \lfloor x \rfloor$
    \item $ \lfloor x + n \rfloor = \lfloor x \rfloor + n$, for some \emph{n}
        $\in \mathbb{Z}$
    \item $ \lceil x + n \rceil = \lceil x \rceil + n$, for some \emph{n} $\in
        \mathbb{Z}$
    \item $(\exists n \in \mathbb{Z})(\exists \epsilon \in \mathbb{R}) \lfloor
        x \rfloor = n + \epsilon$, where $ 0 \leq \epsilon \leq 1$
    \item $f(x) = O(g(x)) \rightarrow (\exists k \in \mathbb{Z})(\exists c \in
        \mathbb{R^{+}})(\forall x > k)[f(x) \leq c \times g(x)]$
    \item $f(x) = \Omega(g(x)) \rightarrow (\exists k \in \mathbb{Z})(\exists c \in
        \mathbb{R^{+}})(\forall x > k)[f(x) \geq c \times g(x)]$
    \item $f(x) = \Theta(g(x)) \rightarrow [f(x) = O(g(x))] \wedge [f(x) =
        \Omega(g(x))]$
\end{symbollist}

\subsection{Number Theory}
\begin{symbollist}
    \item $a / b \rightarrow (\exists c \in \mathbb{Z})[a = b \times c]$
    \item $a / b \wedge a / c \rightarrow a / (b + c)$
    \item $a/ b \wedge b / c \rightarrow a / c $
    \item $\gcd(a, b) = am + bn$, for some $m, n \in \mathbb{Z}$
        (\textbf{Bezout's Identity})
    \item $\gcd(a, b) \times \mathrm{lcm}(a, b) = a \times b$
    \item $a \equiv b[m] \wedge c \equiv d[m] \rightarrow (a + b) \equiv (c +
        d)[m]$
    \item $a \equiv b[m] \wedge c \equiv d[m] \rightarrow (ab) \equiv
        (cd)[m]$
    \item If \emph{p} is a prime number, then $a^{p} \equiv a[p]$
        (\textbf{Fermat's Little Theorem})
    \item If \emph{p} is a prime number and $p / ab$, then $p / a \vee p / b$
        (\textbf{Euclid's Identity})
\end{symbollist}

\subsection{Counting}
\begin{symbollist}
    \item $\mid A \cup B \mid$ =  $\mid A \mid + \mid B \mid - \mid A \cap B
        \mid$
    \item $P(n,r) \frac{n!}{(n - r)!}$ (\textbf{Permutations})
    \item $C(n,r)$ or $\binom{n}{r}$ = $\frac{n!}{r!(n - r)!}$
        (\textbf{Combinations})
\end{symbollist}

\chapter{Propositional Logic}
\begin{definition}
    A \underline{\textbf{proposition}} is a statement with exactly one truth
    value.
\end{definition}
\begin{definition}
    Two propositions are said to be \underline{\textbf{equivalent}} if they
    have the same truth table.
\end{definition}
\begin{definition}
    A \underline{\textbf{tautology}} is a statement that is always true.
\end{definition}
\begin{definition}
    A \underline{\textbf{contradiction}} is a statement that is always false.
\end{definition}

\section{Truth Tables and Logical Equivalences}
If you are given a proposition and asked to check if it is a tautology or a
contradiction, here are two different ways to proceed:

\begin{enumerate}[noitemsep]
    \item Construct a truth table for the given proposition.
    \item Use logical equivalences.
\end{enumerate}

\begin{question}
    Is this a tautology or a contradiction?

    \[
        [p \wedge (q \wedge r)] \rightarrow [( ( (r \wedge p)\wedge q) \vee q)]
    \]
\end{question}

\begin{solution}
    \textbf{1 - Truth Table:} For readability, let us define $\alpha =  p
    \wedge (q \wedge r)$ and let us define $\beta = ( ( (r \wedge p)\wedge q)
    \vee q)$

    \pagebreak[4]

    \begin{table}[ht]
        \caption{Truth Table}
        \vspace{0.5em}
        \centering % used for centering table
        \begin{tabular}{c c c c c c c c c } % 9 columns
        % centered columns (4 columns)
            \toprule
            p & q & r & $q \wedge r$ &  $\alpha$ & $(r \wedge p)$ & $((r \wedge
            p) \wedge q)$ & $\beta$ & $\alpha \rightarrow \beta$\\ [0.5ex]
            \midrule
            T & T & T & T & T & T & T & T & \textbf{T} \\
            T & T & F & F & F & F & F & T & \textbf{T} \\
            T & F & T & F & F & T & F & F & \textbf{T} \\
            T & F & F & F & F & F & F & F & \textbf{T} \\
            F & T & T & T & F & F & F & T & \textbf{T} \\
            F & T & F & F & F & F & F & T & \textbf{T} \\
            F & F & T & F & F & F & F & F & \textbf{T} \\
            F & F & F & F & F & F & F & F & \textbf{T} \\
            \bottomrule
        \end{tabular}
        \label{table:question_1_truthtable}
    \end{table}

    \noindent Since $\alpha \rightarrow \beta$ is always true, this is an example
    of a tautology!
\end{solution}

\begin{solution}
    \textbf{2 - Logical Equivalences: }
    \[
        \begin{split}
            &[p \wedge (q \wedge r)] \rightarrow [( ( (r \wedge p)\wedge q) \vee
            q)] \\
            \iff &\neg[p \wedge q \wedge r] \vee [(  (r \wedge p\wedge q) \vee q)]
            \textbf{ (Definition)} \\
            \iff &\neg (p \wedge q \wedge r) \vee [(p \wedge q \wedge r) \vee q]
            \textbf{ (Commutative law)} \\
            \iff &[\neg (p \wedge q \wedge r) \vee (p \wedge q \wedge r)] \vee
            q \textbf{ (Associative Law)} \\
            \iff &T \vee q \textbf{ (Complement Law)} \\
            \iff &T (\textbf{Identity Law}) \\
        \end{split}
    \]
    \noindent Since the result is always true, the given proposition is a
    tautology! \\
\end{solution}

\begin{remark}
    As you can see, solving problems through logical equivalences is quicker,
    but require you to manipulate the given propositions. If you are
    uncomfortable doing this, please feel free to resort to Truth Tables. The
    same question showed up on my midterm, and I used logical equivalences to
    solve the problem.
\end{remark}

\section{Knights and Knaves}
You can always expect a question on logic puzzles on the midterms and the
finals. These questions are actually fun to do and are not too difficult.

You will be given a situation and you are required to use truth tables to solve
the problem. Here is a sample problem (From a past midterm):

A very special island is inhabited only by Knights and Knaves. Knights always
tell the truth,while Knaves always lie. You meet three inhabitants: Alex, John
and Sally.Alex says, “John is a Knight if and only if Sally is a Knave”. John
says, “If Sally is a Knight, then Alex is a Knight”.

\begin{question}
    Can you find what Alex, John, and Sally are? Explain your answer.
\end{question}

\begin{solution}
    Let us break down the problem:

    You have three people: Alex (A), John (J), and Sally (S). Each of them are
    either a Knight or a Knave. Hence, we have 8 possible rows in our truth table.

    Let us also break down what the people have to say:

    \begin{enumerate}
        \item Alex says, ``John is a Knight if and only if Sally is a Knave,"
            which basically means: \textbf{John is a Knight} $\iff$
            \textbf{Sally is a Knave.}
        \item John says, ``If Sally is a Knight, then Alex is a Knight," which
            basically means: \textbf{Sally is a Knight} $\implies$ \textbf{Alex
            is a Knight.}
    \end{enumerate}

    With this information, let us construct our truth table:

    \begin{table}[ht]
        \caption{Truth Table}
        \centering
        \begin{tabular}{c c c c c c}
            \toprule
            A & J & S & Alex Says & John Says & Does This Work?\\ [0.5ex]
            \midrule
            Knight & Knight & Knight & F & T & \textbf{No} - Alex is a Knight who is
            lying\\
            Knight & Knight & Knave & T & T & \textbf{YES}\\
            Knight & Knave & Knight & T & T & \textbf{No} - John is a Knave who is telling
            the truth\\
            Knight & Knave & Knave & F & T & \textbf{No} -John is a Knave who is telling
            the truth\\
            Knave & Knight & Knight & F & F & \textbf{No} - John is a Knight who is lying\\
            Knave & Knight & Knave & T & T & \textbf{No} - Alex is a Knave who is telling
            the truth\\
            Knave & Knave & Knight & T & F & \textbf{No} - Alex is a Knave who is telling
            the truth \\
            Knave & Knave & Knave & F & T & \textbf{No} - John is a Knave who is telling
            the truth\\
            \bottomrule
        \end{tabular}
        \label{table:nonlin} % is used to refer this table in the text
    \end{table}
    \noindent From this, we can see that there is only one possible
    combination, that Alex and John are Knights and Sally is a Knave. \\ \\
\end{solution}

\begin{remark}
    Sometimes there may be more than one possible combination that works out,
    in that case it is not possible to correctly determine who is who, but it
    is one of those correct combinations.
\end{remark}

\begin{remark}
    There's almost always a Knights and Knaves (or a variation) question on the
    exams.
\end{remark}

\begin{remark}
    If you are interested in these problems, These problems are called
    \emph{Smullyan's Island Puzzles}.
\end{remark}

\section{Additional Exercises}
Please note that there are no solutions for the following questions.

\subsection{Logic}
\begin{question}
    Construct the truth table for the following proposition:

    \[
        [p \wedge (p \rightarrow q)] \rightarrow q
    \]

    This rule of inference is commonly referred to as \emph{Modus Ponens}.
\end{question}

\begin{question}
    Prove or disprove:

    \[
        (p \rightarrow q) \Leftrightarrow (q \rightarrow p)
    \]
\end{question}

\begin{question}
    Is this a tautology or a contradiction?

    \[
        (( (P \rightarrow Q) \wedge (R \rightarrow S) \wedge (P \vee R))) \rightarrow
        (Q \vee S)
    \]

    You may use either Truth Tables or Logical Equivalences for this question.

    This proposition is usually referred to as the \emph{Constructive
    Dilemma}.
\end{question}

\begin{question}
    Prove or disprove:

    \[
        (p \oplus q) \Leftrightarrow (p \wedge \neg q) \vee (\neg p \wedge q)
    \]
\end{question}

\subsection{Logic Puzzles}
%1) You meet two inhabitants, Alex and John, in an island of %Knights and
%Knaves. You also know that Knights always tell %the truth and Knaves always
%lie. Alex tells you, "At least %one of us is a Knave". Again, Alex tells you,
%"Both of us are %Knaves". What can you conclude from the given information? \\
%\\

\begin{question}
    A very special island is inhabited only by Knights and Knaves. Knights
    always tell the truth, while Knaves always lie. You meet three inhabitants:
    Alex, John and Sally. Alex says, "John is a Knight, if and only if Sally is
    a Knave". John says, "If Sally is a Knight, then Alex is a Knight". What
    can you conclude from the given information?
\end{question}

\begin{question}
    A very special island is inhabited only by knights and knaves. Knights
    always tell the truth, and knaves always lie. You meet three inhabitants:
    Alex, John and Sally. Alex says, "At least one of the following is true:
    that Sally is a knave or that I am a knight." John says, "Alex could claim
    that I am a knave." Sally claims, "Neither Alex nor John are knights." What
    can you conclude from the given information?
\end{question}

\chapter{Proofs and Proof Methods}
Writing proofs deepens your knowledge of the subject at hand. ECS 20 is a proof
heavy course and prepares you to write extensive proofs in upper division CS
and Math classes. Almost 60\% of your exam will be based on proofs and will
test your ability to write a flawless proof. Here are three important proof
methods that this chapter will cover:

\begin{itemize}
    \item Direct Proof
    \item Indirect Proof (Proof by Contraposition)
    \item Proof by Contradiction
    \item Proof by Cases
    \item Constructive and Non-Constructive Proofs
\end{itemize}

\section{Direct Proof}
This is the most basic proof method. Here, we will be given a statement of the
form $ P \rightarrow Q $ and you will be asked to prove this propositional
form. Here is how you could prove something directly:

% https://tex.stackexchange.com/questions/108236/enumerate-list-numbers-with-prefix
% TODO make a macro for steps
\begin{enumerate}[label=\textbf{Step \arabic*},leftmargin=*]
    \item Given $P \rightarrow Q$, assume P is true.
    \item Using $P$, try to use the given information of $P$ to arrive at $Q$.
\end{enumerate}

Here is a simple example:

\begin{theorem}
    If $a$ is an even integer, then $a^{3} - 6a$ is even.
\end{theorem}

\begin{proof}
    Let us use a direct proof.

    Assume $a$ is an even integer.

    Then,
    \[
        \begin{split}
            \exists k \in \mathbb{Z}, a &= 2k \\
            a^{3} - 6a &= (2k)^{3} - 6(2k) \\
            &= 8k^{3} - 12k \\
            &= 2(4k^{3} - 6k) \\
            &= 2m\text{, where } m = 4k^{3} - 6k \\
        \end{split}
    \]
    Thus, $a^3 - 6a$ is even. Therefore, if $a$ is even, then $a^{3} -6a$ is even
\end{proof}

\section{Indirect Proof (Proof by Contraposition)}
Suppose we are given a statement of the form $P \rightarrow Q$. Sometimes, a
direct proof might be very hard (sometimes impossible) to go about. Therefore
we resort to another proof method that simplifies things. In this proof, we
will try to show $\neg Q \rightarrow \neg P$ is true. By drawing out a simple
truth table you can show that $ P \rightarrow Q \Leftrightarrow \neg Q
\rightarrow \neg P$.

Here is how you can prove something by contraposition:

\begin{enumerate}[label=\textbf{Step \arabic*},leftmargin=*]
    \item Given $P \rightarrow Q$, Assume $\neg Q$ is true.
    \item Using $\neg Q$, try to arrive at $\neg P$.
\end{enumerate}

\noindent Here is a simple example:

\begin{theorem}
    If $a^{2}$ is even, then $a$ is even
\end{theorem}

\begin{proof}
    Let us use an indirect proof.

    Assume $a$ is odd (Assume $\neg Q$).

    Then,
    \[
        \begin{split}
            \exists k \in \mathbb{Z}, a &= 2k + 1 \\
            a^{2} &= (2k + 1)^{2} = 4k^2 + 4k + 1 \\
            \Leftrightarrow a^{2} &= 2(2k^{2} + 2k) + 1 \\
            \Leftrightarrow &= 2m + 1 \text{, where } m = 2k^{2} + 2k \\
        \end{split}
    \]

    Therefore, $a^{2}$ is odd.  Thus, we have shown that if $a$ is odd, then
    $a^{2}$ is odd.

    Therefore by contraposition, If $a^{2}$ is even then, $a$ is even.
\end{proof}

\section{Proof By Contradiction}
Suppose we are given asked to prove a given statement, Q, a direct proof might
not be feasible. Hence we resort to our last proof method to prove statements
that might look difficult to prove directly. In this proof method, we assume
$\neg Q$ to be true. Suppose you are given a statement of the form $P
\rightarrow Q$, then you assume $\neg (P \rightarrow Q)$, which is basically $P
\wedge \neg Q$.

Here is a simple example:
\begin{theorem}
    If $(a,b) \in \mathbb{Z}^{2}$, then $a^{2} - 4b \neq 2$
\end{theorem}

\begin{proof}
    Let us use a proof by contradiction. Assume $P \wedge \neg Q$ is true. Then
    $(a,b) \in \mathbb{Z}^{2} \wedge (a^{2} -4b = 2)$ is true.

    \[
        \begin{split}
            \text{If }&a^{2} - 4b = 2\text{, then} \\
            &a^{2} = 2 + 4b \\
            \iff &a^{2}\text{ is even} \\
            \iff &a\text{ is even \footnote{You can prove this by
            contraposition. However, If you're interested in using a direct
            proof to show that $a^{2}$ is even $\rightarrow a$ is even, then
            you'd have to wait till we reach number theory ;)}} \\
            \iff &\exists k \in \mathbb{Z}a = 2k \\
            \iff &(2k)^{2} - 4b = 2 \\
            \iff &4k^{2} - 4b = 2 \\
            \iff &2k^{2} - 2b = 1 \\
            \iff &2 \times (k^{2} - 2b) = 1 \\
        \end{split}
    \]
    $\Leftrightarrow$ An even integer is equal to the odd integer 1, which is a
    contradiction.

    This contradiction arises due to our incorrect assumption that $\neg (P
    \rightarrow Q)$ was true.

    Therefore, $P \rightarrow Q$ is true.

    Thus, If $(a,b) \in \mathbb{Z}^{2}$, then $a^{2} - 4b \neq 2$
\end{proof}

\section{Proof by Cases}
This is a proof method that is always used when you do not know the constraints
to a problem. Let us illustrate this with a very simple example:

\begin{theorem}
    If $a$ is an integer, then $2a + 1$ is odd.
\end{theorem}

\begin{proof}
    Here, we do not know if $a$ is an even or odd integer. Thus, we consider
    both cases and try to prove our hypothesis:

    \begin{caselist}
        \item $a$ is even. \\
            Then, ($\exists k \in \mathbb{Z}$)[$a = 2k$]. \\
            Then, $2a + 1 = 2(2k) + 1$, which is odd!
        \item $a$ is odd. \\
            Then, ($\exists k \in \mathbb{Z}$)[$a = 2k + 1$]. \\
            Then, $2a + 1 = 2(2k + 1) + 1$, which is also odd!
    \end{caselist}
\end{proof}

\section{Constructive and Non-Constructive Proofs}
\subsection{Constructive Proofs}
Constructive proofs are proof methods in which a specific example is provided.
These proofs are generally referred to as "\emph{Existence Proofs}".

Let us take two separate examples to illustrate this proof method.

\begin{example}
    Show that there exist two integers a and b such that $a + b$ and $a - b$
    are both prime numbers.
    \begin{proof}
        We only need to show a single pair $(a,b)$ for which $a + b$ and $a - b$
        are both prime numbers.

        Let us take $a = 5$ and $b = 2$.

        Then, $a + b = 7$ and $a - b = 3$.

        Both 7 and 3 are prime numbers.

        Thus, we found a particular pair of numbers for which the hypothesis holds.
    \end{proof}
\end{example}

\begin{example}
    Prove or disprove the following:
    \begin{center}
        For all $k \in \mathbb{Z}$ greater than 2, $2k + 7$ is a prime number.
    \end{center}
    \begin{proof}
        Let us try out a few numbers and see what we get:

        \begin{trylist}
            \item $k = 2$ \\
                Then, $2k + 7 = (2 \times 2) + 7 = 4 + 7 = 11$ which is a prime
                number.
            \item \textbf{$k = 3$} \\
                Then, $2k + 7 = (2 \times 3) + 7 = 6 + 7 = 13$ which is a prime number.
            \item $k = 4$ \\
                Then, $2k + 7 = (2 \times 4) + 7 = 8 + 7 = 15$ \textbf{which is NOT
                prime}.
        \end{trylist}

        \noindent Thus, we have found one example which violates the hypothesis.

        \noindent Thus, the given statement is \textbf{FALSE}!
    \end{proof}
\end{example}

\subsection{Non-Constructive Proofs}
This is another kind of an existence proof. In this method, we do not
explicitly find a value of 'x' such that $P(x)$ is true. Instead, we show that
there must exist some x for which $P(x)$ is true. This might seem like a
confusing definition, so let us use this in an example:

\begin{question}
    Show that there exists a pair of irrational numbers a and b such that $c =
    a^{b}$ is rational.
\end{question}

\begin{remark}
    This is a very famous and interesting problem in the field of proofs and
    discrete mathematics. Let us look at the solution:
\end{remark}

\begin{proof}
    We know that $\sqrt 2$ is irrational. Let us define $c = (\sqrt 2)^{\sqrt
    2}$.

    Additionally, let us define $d = c ^{\sqrt2}$.

    Then, $d = (\sqrt 2) ^{\sqrt 2 \times \sqrt 2} = (\sqrt 2) ^{2} = 2$ which
    is rational.
\end{proof}

Note that we did not explicitly find the values of a and b in the above
example. We simply showed that there could exist a pair of irrational numbers
in which $a^{b}$ could be rational.

\textbf{\emph{Trivia: }} Did you know that there is a constructive
proof for the above mentioned example? It is quite hard to find the right
examples for a and b, but it is possible. I haven't included the answers, but
do think about it!

\chapter{Set Theory}
Set Theory is one of the most fundamental topics in Abstract/Discrete
Mathematics. If you have ever heard about Relations, Equivalence Classes,
Groups, Rings, and Fields, they are all based off Set Theory! This chapter
covers the following and it is essential that you know all of it:

\section{Definitions}
\begin{definition}
    Set A is said to be a \underline{\textbf{subset}} of the set B if ($\forall
    x \in A$)[$x \in B$].
\end{definition}

\begin{definition}
    The \underline{\textbf{union}} of set A and set B is the set: \{$x \mid x
    \in A \vee x \in B$\}.
\end{definition}

\begin{definition}
    The \underline{\textbf{intersection}} of A and B is the set: \{$x \mid x
    \in A \wedge x \in B$\}.
\end{definition}

\begin{definition}
    The \underline{\textbf{difference}} of A and B is the set: \{$x \mid x \in
    A \wedge x \not\in B$\}.
\end{definition}

\begin{definition}
    Sets A and B are said to be \underline{\textbf{equal}} if $(A \subset B)
    \wedge (B \subset A)$.
\end{definition}

\begin{definition}
    The \underline{\textbf{complement}} of A is the set: \{$x \mid x \in D
    \wedge x \not\in A$\}, where D is the domain or universal set.
\end{definition}

\begin{definition}
    The \underline{\textbf{cardinality}} of A is the number of elements in A.
\end{definition}

\section{Proofs}
This class in particular will not test too many hard proofs in Set Theory. In
fact, most proofs in this chapter will require you to construct simple direct
proofs. So, let us show you an example of how proofs in set theory might look
like:

\begin{question}
    If A and B are two sets, show that $\overline{A \cap B} = \overline{A} \cup
    \overline{B}$.
\end{question}

\begin{proof}
    We need to show that $\overline{A \cap B} \subset \overline{A} \cup
    \overline{B}$ and $\overline{A} \cup \overline{B} \subset \overline{A \cap
    B}$

    Let $x$ be an arbitrary element $\in \overline{A \cap B}$
    \[
        \begin{split}
            \implies &x \not\in A \cap B \\
            \implies &\neg(x \in A \wedge x \in B) \\
            \implies &x \not\in A \vee x \not\in B \\
            \implies &x \in \overline{A} \cup \overline{B} \\
            \implies &\overline{A \cap B} \subset \overline{A} \cup \overline{B}
        \end{split}
    \]

    Let $x \in \overline{A} \cup \overline{B}$

    \[
        \begin{split}
            \implies &x \in \overline{A} \vee x \in \overline{B} \\
            \implies &\neg x \in A \vee \neg x \in B \\
            \implies &\neg (x \in A \wedge x \in B) \\
            \implies &\neg (x \in A \cap B) \\
            \implies &x \in \overline{A \cap B} \\
            \implies &\overline{A} \cup \overline{B} \subset \overline{A \cap B}
        \end{split}
    \]

    Thus, $\overline{A \cap B} = \overline{A} \cup \overline{B}$.
\end{proof}

You also need to know set identities but they are the same as the basic
identities of propositional logic, so I haven't included them in this section.

I have also skipped the section of Generalized Unions and Intersections as they
are based on the basic definition of union and intersection of sets. Otherwise,
that is pretty much it with regards to set theory!

\chapter{Functions}
\section{Basics}
A function \emph{f}, from A to B is an assignment of exactly one element of B
to each element in A. \\
For example: Let \emph{f} be a function such that $\forall x \in \mathbb{R}$,
$f(x) = x^{2}$. You can observe that for any desired value of x, there is only
one unique mapping to $f(x)$. The reverse is not always true! \\ \\
A function is considered \textbf{\underline{one-to-one or injective}} if $f(x)
= f(y) \Leftrightarrow x = y$ \\ \\
A function from A to B is \textbf{\underline{onto or surjective}} if $\forall y
\in B$ $\exists x \in A$ such that $f(x) = y$. \\ \\
A function that is both one-to-one and onto is called a
\textbf{\underline{bijection}}. A function that is bijective implies that an
inverse for the function exists! \\ \\
\textsc{\underline{For Example:}} Consider the function $f:\mathbb{R}
\rightarrow \mathbb{R}$ such that $f(x) = 12x + 5$. Show that \emph{f} has an
inverse and find its inverse. \\
\indent \textbf{Solution:} First we have to show that f is injective. \\
\indent \indent Let us assume that $f(x_{1}) = f(x_{2})$.\\ \indent \indent
Then, $12x_{1} + 5 = 12x_{2} + 5 \Leftrightarrow 12x_{1} = 12{x2}
\Leftrightarrow x_{1} = x_{2}$. Thus, f is injective! \\
\indent Next, we have to show that f is surjective. \\
\indent \indent Let $f(x) = y$. Then, $y = 12x + 5$. \\
\indent \indent Then, $x = \frac{y - 5}{12}$. One can observe that for any real
number y, there exists a real number x such that $x = \frac{y - 5}{12}$. Thus,
f is surjective! \\
\indent Since f is a bijection, the inverse of f $f^{-1}(x) = \frac{x - 5}{12}$

\section{Floor and Ceiling Functions}
The \textbf{floor} of any real number returns the greatest integer that is less
than or equal to the real number. \\
The \textbf{ceiling} of any real number returns the smallest integer that is
greater than or equal to the real number. \\

For example: $\lfloor 2.5 \rfloor = 2$, $\lfloor -3.4 \rfloor = -4, \lfloor 7
\rfloor = 7$ \\
\indent \indent $\lceil 2.5 \rceil = 3, \lceil -3.4 \rceil = -3, \lceil \pi
\rceil = 4$.

\subsection{Proofs regarding Floor and ceiling functions}
I think it would be quite redundant to show proofs regarding floor and ceiling
functions. Questions that are similar to the ones shown in the
\href{http://nook.cs.ucdavis.edu/~koehl/Teaching/ECS20/Lectures/Lecture5_notes.pdf}{\underline{\emph{lecture
notes}}} generally show up on the midterms/final. Thus, I have omitted this one
topic as you can just read it up from the notes.

\section{Growth of Functions}
Often times, your task would be to determine the running time of an algorithm
that you just developed. These running times are represented as $O()$ (Big Oh),
$\Omega()$ (Big Omega), and $\Theta()$ (Big Theta) respectively.

\subsection{Big-O Notation}
The Big-O Notation is often used to represent the worst case analysis. Think of
Big-O as an upper bound for functions. The function f(x) is O(g(x)) if $[f(x)
\leq c \times g(x)]$ for all x > k, where c is a positive constant.\\
Let us consider an example: \\
Show that $x^{2} + 5x + 3$ is $O(x^{2})$ \\
\indent \emph{Solution:} In order for $x^{2} + 5x + 3$ to be $O(x^{2})$, $x^{2}
+ 5x + 3 \leq cx^{2}$. \\
\indent \indent We know that if $x > 1$ then, $5x \leq 5x^{2}$ and $3 \leq
3x^{2}$. \\
\indent \indent Adding these up we get: $x^{2} + 5x + 3 \leq x^{2} + 5x^{2} +
3x^{2} \leq 9x^{2}$. \\
\indent \indent Thus by selectively choosing $k = 1$ and $c = 9$, we can show
that $x^{2} + 5x + 3$ is $O(x^{2})$. \\ \\
\noindent For further verification and understanding, let us graph the
functions $x^{2} + 5x + 3$ and $9x^{2}$.

\begin{tikzpicture}
\begin{axis}[
    axis lines = left,
    xlabel = $x$,
    ylabel = {$f(x)$},
]
%Below the red parabola is defined
\addplot [
    domain=0:50,
    samples=100,
    color=red,
]
{x^2 + 5*x + 3};
\addlegendentry{$x^2 + 5x + 3$}
%Here the blue parabola is defined
\addplot [
    domain=0:50,
    samples=100,
    color=blue,
    ]
    {9*x^2};
\addlegendentry{$9x^{2}$}

\end{axis}
\end{tikzpicture} \\
As you can see, $x^2 + 5x + 3$ is never greater than $9x^{2}$.

\subsection{Big-Omega Notation}
The Big-Omega Notation is often used to represent the best case analysis. Think
of Big-Omega as an lower bound for functions. The function f(x) is
$\Omega(g(x))$ if $[f(x) \geq c \times g(x)]$ for all x > k, where c is a
positive constant.\\
Let us consider an example: \\
Show that $x^{3} + 6x^{2} + 3$ is $\Omega(x^{3})$ \\
\indent \emph{Solution:} In order for $x^{3} + 6x^{2} + 3$ to be
$\Omega(x^{3})$, $x^{3} + 6x^{2} + 3 \geq cx^{3}$. \\
\indent \indent We know that if $x > 1$ then, $x^{3} + 6x^{2} + 3 \geq x^{3}$.
\\
\indent \indent Thus by selectively choosing $k = 1$ and $c = 1$, we can show
that $x^{3} + 6x^{2} + 3$ is $\Omega(x^{3})$. \\ \\
\noindent For further verification and understanding, let us graph the
functions $x^{3} + 6x^{2} + 3$ and $x^{3}$.

\begin{tikzpicture}
\begin{axis}[
    axis lines = left,
    xlabel = $x$,
    ylabel = {$f(x)$},
]
%Below the red parabola is defined
\addplot [
    domain=0:50,
    samples=100,
    color=red,
]
{x^3 + 6*x^2 + 3};
\addlegendentry{$x^{3} + 6x^{2} + 3$}
%Here the blue parabola is defined
\addplot [
    domain=0:50,
    samples=100,
    color=blue,
    ]
    {x^3};
\addlegendentry{$x^{3}$}

\end{axis}
\end{tikzpicture} \\
As you can see, $x^{3} + 6x^{2} + 3$ is always greater than $x^{3}$.
\subsection{Big-Theta Notation}
The Big-Theta Notation is often used to represent the average case analysis.
Think of Big-Theta as a tight bound for functions. The function f(x) is
$\Theta(g(x))$ if f(x) is $O(g(x))$ and if f(x) is $\Omega(g(x))$ \\
Let us consider an example: \\
Show that $x^{2} + 4x + 7$ is $\Theta(x^{2})$ \\
\indent \emph{Solution:} First, we have to show that $x^{2} + 4x + 7$ is
$O(x^{2})$. \\
\indent \indent Then, $x^{2} + 4x + 7 \leq cx^{2}$. \\
\indent \indent We know that for all $x > 1$, $4x \leq 4x^{2}$ and $7 \leq
7x^{2}$. Adding these up, we can say that $x^{2} + 4x + 7 \leq 12x^{2}$. Thus
by selectively choosing $k = 1$ and $c = 12$ we can conclude that $x^{2} + 4x +
7$ is $O(x^{2})$.\\
\noindent Next, we have to show the Big-Omega relation between these two
functions. \\
\indent \indent We know that for all $x > 1$ $x^{2} + 4x + 7 \geq x^{2}$
\indent \indent Thus by selectively choosing $k = 1$ and $c = 1$ we can
conclude that $x > 1$ $x^{2} + 4x + 7$ is $\Omega(x^{2})$. \\
\indent \indent Thus $\forall x > 1$ we can infer that:
\begin{center}
    $x^{2} \leq x^{2} + 4x + 7 \leq 12x^{2}$
\end{center}
\indent \indent Thus, $x^{2} + 4x + 7$ is $\Theta(x^{2})$
\subsection{Why do we need all this?}
Often times in algorithm analysis, you deal with large chunks of data. If, for
example, $n = 10$ where n is the number of elements, then it doesn't really
matter how good or bad our Big-Oh is. What if $n = 1,000,000$? Then $n^{2}$ and
$n^{3}$ would be significantly different. This is why we study the growth of
these functions. These notations are also sometimes referred to as time
complexities. You will learn about time complexities extensively in ECS 60 and
ECS 122A.

\chapter{Number Theory}


\end{document}
