\chapter{Propositional Logic}
\begin{definition}
    A \underline{\textbf{proposition}} is a statement with exactly one truth
    value.
\end{definition}
\begin{definition}
    Two propositions are said to be \underline{\textbf{equivalent}} if they
    have the same truth table.
\end{definition}
\begin{definition}
    A \underline{\textbf{tautology}} is a statement that is always true.
\end{definition}
\begin{definition}
    A \underline{\textbf{contradiction}} is a statement that is always false.
\end{definition}

\section{Truth Tables and Logical Equivalences}
If you are given a proposition and asked to check if it is a tautology or a
contradiction, here are two different ways to proceed:

\begin{enumerate}[noitemsep]
    \item Construct a truth table for the given proposition.
    \item Use logical equivalences.
\end{enumerate}

\begin{question}
    Is this a tautology or a contradiction?

    \[
        [p \wedge (q \wedge r)] \rightarrow [( ( (r \wedge p)\wedge q) \vee q)]
    \]
\end{question}

\begin{solution}
    \textbf{1 - Truth Table:} For readability, let us define $\alpha =  p
    \wedge (q \wedge r)$ and let us define $\beta = ( ( (r \wedge p)\wedge q)
    \vee q)$

    \pagebreak[4]

    \begin{table}[ht]
        \caption{Truth Table}
        \vspace{0.5em}
        \centering % used for centering table
        \begin{tabular}{c c c c c c c c c } % 9 columns
        % centered columns (4 columns)
            \toprule
            p & q & r & $q \wedge r$ &  $\alpha$ & $(r \wedge p)$ & $((r \wedge
            p) \wedge q)$ & $\beta$ & $\alpha \rightarrow \beta$\\ [0.5ex]
            \midrule
            T & T & T & T & T & T & T & T & \textbf{T} \\
            T & T & F & F & F & F & F & T & \textbf{T} \\
            T & F & T & F & F & T & F & F & \textbf{T} \\
            T & F & F & F & F & F & F & F & \textbf{T} \\
            F & T & T & T & F & F & F & T & \textbf{T} \\
            F & T & F & F & F & F & F & T & \textbf{T} \\
            F & F & T & F & F & F & F & F & \textbf{T} \\
            F & F & F & F & F & F & F & F & \textbf{T} \\
            \bottomrule
        \end{tabular}
        \label{table:question_1_truthtable}
    \end{table}

    \noindent Since $\alpha \rightarrow \beta$ is always true, this is an example
    of a tautology!
\end{solution}

\begin{solution}
    \textbf{2 - Logical Equivalences: }
    \[
        \begin{split}
            &[p \wedge (q \wedge r)] \rightarrow [( ( (r \wedge p)\wedge q) \vee
            q)] \\
            \iff &\neg[p \wedge q \wedge r] \vee [(  (r \wedge p\wedge q) \vee q)]
            \textbf{ (Definition)} \\
            \iff &\neg (p \wedge q \wedge r) \vee [(p \wedge q \wedge r) \vee q]
            \textbf{ (Commutative law)} \\
            \iff &[\neg (p \wedge q \wedge r) \vee (p \wedge q \wedge r)] \vee
            q \textbf{ (Associative Law)} \\
            \iff &T \vee q \textbf{ (Complement Law)} \\
            \iff &T (\textbf{Identity Law}) \\
        \end{split}
    \]
    \noindent Since the result is always true, the given proposition is a
    tautology! \\
\end{solution}

\begin{remark}
    As you can see, solving problems through logical equivalences is quicker,
    but require you to manipulate the given propositions. If you are
    uncomfortable doing this, please feel free to resort to Truth Tables. The
    same question showed up on my midterm, and I used logical equivalences to
    solve the problem.
\end{remark}

\section{Knights and Knaves}
You can always expect a question on logic puzzles on the midterms and the
finals. These questions are actually fun to do and are not too difficult.

You will be given a situation and you are required to use truth tables to solve
the problem. Here is a sample problem (From a past midterm):

A very special island is inhabited only by Knights and Knaves. Knights always
tell the truth,while Knaves always lie. You meet three inhabitants: Alex, John
and Sally.Alex says, “John is a Knight if and only if Sally is a Knave”. John
says, “If Sally is a Knight, then Alex is a Knight”.

\begin{question}
    Can you find what Alex, John, and Sally are? Explain your answer.
\end{question}

\begin{solution}
    Let us break down the problem:

    You have three people: Alex (A), John (J), and Sally (S). Each of them are
    either a Knight or a Knave. Hence, we have 8 possible rows in our truth table.

    Let us also break down what the people have to say:

    \begin{enumerate}
        \item Alex says, ``John is a Knight if and only if Sally is a Knave,"
            which basically means: \textbf{John is a Knight} $\iff$
            \textbf{Sally is a Knave.}
        \item John says, ``If Sally is a Knight, then Alex is a Knight," which
            basically means: \textbf{Sally is a Knight} $\implies$ \textbf{Alex
            is a Knight.}
    \end{enumerate}

    With this information, let us construct our truth table:

    \begin{table}[ht]
        \caption{Truth Table}
        \centering
        \begin{tabular}{c c c c c c}
            \toprule
            A & J & S & Alex Says & John Says & Does This Work?\\ [0.5ex]
            \midrule
            Knight & Knight & Knight & F & T & \textbf{No} - Alex is a Knight who is
            lying\\
            Knight & Knight & Knave & T & T & \textbf{YES}\\
            Knight & Knave & Knight & T & T & \textbf{No} - John is a Knave who is telling
            the truth\\
            Knight & Knave & Knave & F & T & \textbf{No} -John is a Knave who is telling
            the truth\\
            Knave & Knight & Knight & F & F & \textbf{No} - John is a Knight who is lying\\
            Knave & Knight & Knave & T & T & \textbf{No} - Alex is a Knave who is telling
            the truth\\
            Knave & Knave & Knight & T & F & \textbf{No} - Alex is a Knave who is telling
            the truth \\
            Knave & Knave & Knave & F & T & \textbf{No} - John is a Knave who is telling
            the truth\\
            \bottomrule
        \end{tabular}
        \label{table:nonlin} % is used to refer this table in the text
    \end{table}
    \noindent From this, we can see that there is only one possible
    combination, that Alex and John are Knights and Sally is a Knave. \\ \\
\end{solution}

\begin{remark}
    Sometimes there may be more than one possible combination that works out,
    in that case it is not possible to correctly determine who is who, but it
    is one of those correct combinations.
\end{remark}

\begin{remark}
    There's almost always a Knights and Knaves (or a variation) question on the
    exams.
\end{remark}

\begin{remark}
    If you are interested in these problems, These problems are called
    \emph{Smullyan's Island Puzzles}.
\end{remark}

\section{Additional Exercises}
Please note that there are no solutions for the following questions.

\subsection{Logic}
\begin{question}
    Construct the truth table for the following proposition:

    \[
        [p \wedge (p \rightarrow q)] \rightarrow q
    \]

    This rule of inference is commonly referred to as \emph{Modus Ponens}.
\end{question}

\begin{question}
    Prove or disprove:

    \[
        (p \rightarrow q) \Leftrightarrow (q \rightarrow p)
    \]
\end{question}

\begin{question}
    Is this a tautology or a contradiction?

    \[
        (( (P \rightarrow Q) \wedge (R \rightarrow S) \wedge (P \vee R))) \rightarrow
        (Q \vee S)
    \]

    You may use either Truth Tables or Logical Equivalences for this question.

    This proposition is usually referred to as the \emph{Constructive
    Dilemma}.
\end{question}

\begin{question}
    Prove or disprove:

    \[
        (p \oplus q) \Leftrightarrow (p \wedge \neg q) \vee (\neg p \wedge q)
    \]
\end{question}

\subsection{Logic Puzzles}
%1) You meet two inhabitants, Alex and John, in an island of %Knights and
%Knaves. You also know that Knights always tell %the truth and Knaves always
%lie. Alex tells you, "At least %one of us is a Knave". Again, Alex tells you,
%"Both of us are %Knaves". What can you conclude from the given information? \\
%\\

\begin{question}
    A very special island is inhabited only by Knights and Knaves. Knights
    always tell the truth, while Knaves always lie. You meet three inhabitants:
    Alex, John and Sally. Alex says, "John is a Knight, if and only if Sally is
    a Knave". John says, "If Sally is a Knight, then Alex is a Knight". What
    can you conclude from the given information?
\end{question}

\begin{question}
    A very special island is inhabited only by knights and knaves. Knights
    always tell the truth, and knaves always lie. You meet three inhabitants:
    Alex, John and Sally. Alex says, "At least one of the following is true:
    that Sally is a knave or that I am a knight." John says, "Alex could claim
    that I am a knave." Sally claims, "Neither Alex nor John are knights." What
    can you conclude from the given information?
\end{question}
