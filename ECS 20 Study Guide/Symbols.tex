\chapter{Math Symbols, Notations, and Identities}

\section{Symbols and Notations}
Here are the symbols and notations that you must absolutely know!

\subsection{Sets of Numbers}
\begin{symbollist}
    \item[$\mathbb{N}$] = Set of all Natural Numbers.
    \item[$\mathbb{Z}$] = Set of all Integers.
    \item[$\mathbb{Q}$] = Set of all Rational Numbers.
    \item[$\mathbb{R}$] = Set of all Real Numbers.
    \item[$\mathbb{C}$] = Set of all Complex Numbers.
\end{symbollist}

\subsection{Propositional Logic}
\begin{symbollist}
    \item[$\neg P$] = Negation of the proposition, \emph{P}.
    \item[$P \wedge Q$] = Conjunction of the propositions \emph{P and Q}.
    \item[$P \vee Q$] = Disjunction of the propositions \emph{P and Q}.
    \item[$P \oplus Q$] = P XOR Q (Exclusive Or).
    \item[$P \rightarrow Q$] = "If P, then Q" (Conditional).
    \item[$P \leftrightarrow Q$] = "P if and only if Q" (Biconditional).
    \item[$P \Leftrightarrow Q$] = P is equivalent to Q.
\end{symbollist}

% This would prevent LaTeX from "spreading out" the text to fill up the whole
% page, which I personally prefer, but isn't necessary

% \vfill

\subsection{Set Theory}
\begin{symbollist}
    \item[$\emptyset$] = Empty Set.
    \item[$ x \in \mathbb{Q}$] = x belongs to a rational number.
    \item[$A \subset B$] = Set A is a subset of set B.
    \item[$S = \{1,2,3,4\}$] =  An example of a set in roster form.
    \item[$S = \{x \mid 1 \leq x \leq 4, x \in \mathbb{Z}\}$] = An example of a
        set in set-builder form.
    \item[$A \cup B$] = A union B.
    \item[$A \cap B$] = A intersection B.
    \item[$A - B$] = A minus B (Set difference).
    \item[$\overline{A}$] = Complement of A.
    \item[$\mid A \mid$] = Number of elements in A (Cardinality).
\end{symbollist}


\subsection{Functions}
\begin{symbollist}
    \item[$\lceil x \rceil $] = Ceiling of x.
    \item[$\lfloor x \rfloor$] = Floor of x.
    \item[$O(f(x))$] = Big-O of the function \emph{f}.
    \item[$\Omega(f(x))$] = Big-Omega of the function \emph{f}.
    \item[$\Theta(f(x))$] = Big-Theta of the function \emph{f}.
\end{symbollist}

\subsection{Number Theory}
\begin{symbollist}
    \item[$a / b$] = a divides b.
    \item[$a \equiv b \lbrack m \rbrack$] = a is congruent to b modulo m.
\end{symbollist}

\subsection{Miscellaneous}
\begin{symbollist}
    \item[$\sum_{i = 1} ^ {n} i$] = Sum of first n terms.
    \item[$\prod_{i = 1} ^ {n} i$] = Product of first n terms.  $n!$ = \emph{n}
        factorial.
    \item[$\forall x \in \mathbb{Z}, P(x)$] = For all x in the set of integers,
        P(x) is true.
    \item[$\exists x \in \mathbb{Z}, P(x)$] = There exists an x such that P(x)
        is true.
\end{symbollist}

\pagebreak[4]

\section{Important Identities}
\indent \indent I have compiled a bunch of identities that are going to prove
to be very useful for proofs or problem solving. Again, I have broken them down
into chapters for better reference \footnote{I haven't included basic
conjunctions and disjunctions of propositions because you should be knowing
them by now!}.

\subsection{Simple Mathematical Identities}
\begin{symbollist}
    \item $(a + b)^{2} = a^{2} + 2ab + b^{2}$
    \item $(a - b)^{2} = a^{2} - 2ab + b^{2}$
    \item $a^{2} - b^{2} = (a + b)\times(a - b)$
    \item $a^{m} \times a^{n} = a^{m + n}$
    \item $(a^{m})^{n} = a^{mn}$
    \item $\log(a \times b) = \log(a) + \log(b)$
    \item $\log(a^{b}) = b \times \log(a)$
\end{symbollist}

\subsection{Propositional Logic}
\begin{symbollist}
    \item $\neg (\neg P) = P$
    \item $P \oplus Q = (P \vee Q) \wedge (\neg(P \wedge Q))$
    \item $\neg(P \wedge Q) = \neg P \vee \neg Q$
    \item $\neg(P \vee Q) = \neg P \wedge \neg Q$
    \item $P \vee P = P$
    \item $P \wedge P = P$
    \item $P \wedge Q = Q \wedge P$
    \item $P \vee Q = Q \vee P$
    \item $(P \wedge Q) \wedge R = P \wedge (Q \wedge R)$
    \item $(P \vee Q) \vee R = P \vee (Q \vee R)$
    \item $P \wedge (Q \vee R) = (P \vee Q) \wedge (P \vee R)$
    \item $P \vee (Q \wedge R) = (P \wedge R) \vee (P \wedge R)$
    \item $P \leftrightarrow Q \Leftrightarrow (P \rightarrow Q) \wedge (Q
    \rightarrow P)$
    \item $P \rightarrow Q = \neg P \vee Q$
\end{symbollist}

\subsection{Functions}
\begin{symbollist}
    \item $ x - 1 < \lfloor x \rfloor \leq x \leq \lceil x \rceil < x + 1$
    \item $ \lfloor - x \rfloor = - \lceil x \rceil$
    \item $ \lceil -x \rceil = - \lfloor x \rfloor$
    \item $ \lfloor x + n \rfloor = \lfloor x \rfloor + n$, for some \emph{n}
        $\in \mathbb{Z}$
    \item $ \lceil x + n \rceil = \lceil x \rceil + n$, for some \emph{n} $\in
        \mathbb{Z}$
    \item $(\exists n \in \mathbb{Z})(\exists \epsilon \in \mathbb{R}) \lfloor
        x \rfloor = n + \epsilon$, where $ 0 \leq \epsilon \leq 1$
    \item $f(x) = O(g(x)) \rightarrow (\exists k \in \mathbb{Z})(\exists c \in
        \mathbb{R^{+}})(\forall x > k)[f(x) \leq c \times g(x)]$
    \item $f(x) = \Omega(g(x)) \rightarrow (\exists k \in \mathbb{Z})(\exists c \in
        \mathbb{R^{+}})(\forall x > k)[f(x) \geq c \times g(x)]$
    \item $f(x) = \Theta(g(x)) \rightarrow [f(x) = O(g(x))] \wedge [f(x) =
        \Omega(g(x))]$
\end{symbollist}

\subsection{Number Theory}
\begin{symbollist}
    \item $a / b \rightarrow (\exists c \in \mathbb{Z})[a = b \times c]$
    \item $a / b \wedge a / c \rightarrow a / (b + c)$
    \item $a/ b \wedge b / c \rightarrow a / c $
    \item $\gcd(a, b) = am + bn$, for some $m, n \in \mathbb{Z}$
        (\textbf{Bezout's Identity})
    \item $\gcd(a, b) \times \mathrm{lcm}(a, b) = a \times b$
    \item $a \equiv b[m] \wedge c \equiv d[m] \rightarrow (a + b) \equiv (c +
        d)[m]$
    \item $a \equiv b[m] \wedge c \equiv d[m] \rightarrow (ab) \equiv
        (cd)[m]$
    \item If \emph{p} is a prime number, then $a^{p} \equiv a[p]$
        (\textbf{Fermat's Little Theorem})
    \item If \emph{p} is a prime number and $p / ab$, then $p / a \vee p / b$
        (\textbf{Euclid's Identity})
\end{symbollist}

\subsection{Counting}
\begin{symbollist}
    \item $\mid A \cup B \mid$ =  $\mid A \mid + \mid B \mid - \mid A \cap B
        \mid$
    \item $P(n,r) \frac{n!}{(n - r)!}$ (\textbf{Permutations})
    \item $C(n,r)$ or $\binom{n}{r}$ = $\frac{n!}{r!(n - r)!}$
        (\textbf{Combinations})
\end{symbollist}
