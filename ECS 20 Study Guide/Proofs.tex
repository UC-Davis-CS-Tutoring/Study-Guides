\chapter{Proofs and Proof Methods}
Writing proofs deepens your knowledge of the subject at hand. ECS 20 is a proof
heavy course and prepares you to write extensive proofs in upper division CS
and Math classes. Almost 60\% of your exam will be based on proofs and will
test your ability to write a flawless proof. Here are three important proof
methods that this chapter will cover:

\begin{itemize}
    \item Direct Proof
    \item Indirect Proof (Proof by Contraposition)
    \item Proof by Contradiction
    \item Proof by Cases
    \item Constructive and Non-Constructive Proofs
\end{itemize}

\section{Direct Proof}
This is the most basic proof method. Here, we will be given a statement of the
form $ P \rightarrow Q $ and you will be asked to prove this propositional
form. Here is how you could prove something directly:

% https://tex.stackexchange.com/questions/108236/enumerate-list-numbers-with-prefix
% TODO make a macro for steps
\begin{enumerate}[label=\textbf{Step \arabic*},leftmargin=*]
    \item Given $P \rightarrow Q$, assume P is true.
    \item Using $P$, try to use the given information of $P$ to arrive at $Q$.
\end{enumerate}

Here is a simple example:

\begin{theorem}
    If $a$ is an even integer, then $a^{3} - 6a$ is even.
\end{theorem}

\begin{proof}
    Let us use a direct proof.

    Assume $a$ is an even integer.

    Then,
    \[
        \begin{split}
            \exists k \in \mathbb{Z}, a &= 2k \\
            a^{3} - 6a &= (2k)^{3} - 6(2k) \\
            &= 8k^{3} - 12k \\
            &= 2(4k^{3} - 6k) \\
            &= 2m\text{, where } m = 4k^{3} - 6k \\
        \end{split}
    \]
    Thus, $a^3 - 6a$ is even. Therefore, if $a$ is even, then $a^{3} -6a$ is even
\end{proof}

\section{Indirect Proof (Proof by Contraposition)}
Suppose we are given a statement of the form $P \rightarrow Q$. Sometimes, a
direct proof might be very hard (sometimes impossible) to go about. Therefore
we resort to another proof method that simplifies things. In this proof, we
will try to show $\neg Q \rightarrow \neg P$ is true. By drawing out a simple
truth table you can show that $ P \rightarrow Q \Leftrightarrow \neg Q
\rightarrow \neg P$.

Here is how you can prove something by contraposition:

\begin{enumerate}[label=\textbf{Step \arabic*},leftmargin=*]
    \item Given $P \rightarrow Q$, Assume $\neg Q$ is true.
    \item Using $\neg Q$, try to arrive at $\neg P$.
\end{enumerate}

\noindent Here is a simple example:

\begin{theorem}
    If $a^{2}$ is even, then $a$ is even
\end{theorem}

\begin{proof}
    Let us use an indirect proof.

    Assume $a$ is odd (Assume $\neg Q$).

    Then,
    \[
        \begin{split}
            \exists k \in \mathbb{Z}, a &= 2k + 1 \\
            a^{2} &= (2k + 1)^{2} = 4k^2 + 4k + 1 \\
            \Leftrightarrow a^{2} &= 2(2k^{2} + 2k) + 1 \\
            \Leftrightarrow &= 2m + 1 \text{, where } m = 2k^{2} + 2k \\
        \end{split}
    \]

    Therefore, $a^{2}$ is odd.  Thus, we have shown that if $a$ is odd, then
    $a^{2}$ is odd.

    Therefore by contraposition, If $a^{2}$ is even then, $a$ is even.
\end{proof}

\section{Proof By Contradiction}
Suppose we are given asked to prove a given statement, Q, a direct proof might
not be feasible. Hence we resort to our last proof method to prove statements
that might look difficult to prove directly. In this proof method, we assume
$\neg Q$ to be true. Suppose you are given a statement of the form $P
\rightarrow Q$, then you assume $\neg (P \rightarrow Q)$, which is basically $P
\wedge \neg Q$.

Here is a simple example:
\begin{theorem}
    If $(a,b) \in \mathbb{Z}^{2}$, then $a^{2} - 4b \neq 2$
\end{theorem}

\begin{proof}
    Let us use a proof by contradiction. Assume $P \wedge \neg Q$ is true. Then
    $(a,b) \in \mathbb{Z}^{2} \wedge (a^{2} -4b = 2)$ is true.

    \[
        \begin{split}
            \text{If }&a^{2} - 4b = 2\text{, then} \\
            &a^{2} = 2 + 4b \\
            \iff &a^{2}\text{ is even} \\
            \iff &a\text{ is even \footnote{You can prove this by
            contraposition. However, If you're interested in using a direct
            proof to show that $a^{2}$ is even $\rightarrow a$ is even, then
            you'd have to wait till we reach number theory ;)}} \\
            \iff &\exists k \in \mathbb{Z}a = 2k \\
            \iff &(2k)^{2} - 4b = 2 \\
            \iff &4k^{2} - 4b = 2 \\
            \iff &2k^{2} - 2b = 1 \\
            \iff &2 \times (k^{2} - 2b) = 1 \\
        \end{split}
    \]
    $\Leftrightarrow$ An even integer is equal to the odd integer 1, which is a
    contradiction.

    This contradiction arises due to our incorrect assumption that $\neg (P
    \rightarrow Q)$ was true.

    Therefore, $P \rightarrow Q$ is true.

    Thus, If $(a,b) \in \mathbb{Z}^{2}$, then $a^{2} - 4b \neq 2$
\end{proof}

\section{Proof by Cases}
This is a proof method that is always used when you do not know the constraints
to a problem. Let us illustrate this with a very simple example:

\begin{theorem}
    If $a$ is an integer, then $2a + 1$ is odd.
\end{theorem}

\begin{proof}
    Here, we do not know if $a$ is an even or odd integer. Thus, we consider
    both cases and try to prove our hypothesis:

    \begin{caselist}
        \item $a$ is even. \\
            Then, ($\exists k \in \mathbb{Z}$)[$a = 2k$]. \\
            Then, $2a + 1 = 2(2k) + 1$, which is odd!
        \item $a$ is odd. \\
            Then, ($\exists k \in \mathbb{Z}$)[$a = 2k + 1$]. \\
            Then, $2a + 1 = 2(2k + 1) + 1$, which is also odd!
    \end{caselist}
\end{proof}

\section{Constructive and Non-Constructive Proofs}
\subsection{Constructive Proofs}
Constructive proofs are proof methods in which a specific example is provided.
These proofs are generally referred to as "\emph{Existence Proofs}".

Let us take two separate examples to illustrate this proof method.

\begin{example}
    Show that there exist two integers a and b such that $a + b$ and $a - b$
    are both prime numbers.
    \begin{proof}
        We only need to show a single pair $(a,b)$ for which $a + b$ and $a - b$
        are both prime numbers.

        Let us take $a = 5$ and $b = 2$.

        Then, $a + b = 7$ and $a - b = 3$.

        Both 7 and 3 are prime numbers.

        Thus, we found a particular pair of numbers for which the hypothesis holds.
    \end{proof}
\end{example}

\begin{example}
    Prove or disprove the following:
    \begin{center}
        For all $k \in \mathbb{Z}$ greater than 2, $2k + 7$ is a prime number.
    \end{center}
    \begin{proof}
        Let us try out a few numbers and see what we get:

        \begin{trylist}
            \item $k = 2$ \\
                Then, $2k + 7 = (2 \times 2) + 7 = 4 + 7 = 11$ which is a prime
                number.
            \item \textbf{$k = 3$} \\
                Then, $2k + 7 = (2 \times 3) + 7 = 6 + 7 = 13$ which is a prime number.
            \item $k = 4$ \\
                Then, $2k + 7 = (2 \times 4) + 7 = 8 + 7 = 15$ \textbf{which is NOT
                prime}.
        \end{trylist}

        \noindent Thus, we have found one example which violates the hypothesis.

        \noindent Thus, the given statement is \textbf{FALSE}!
    \end{proof}
\end{example}

\subsection{Non-Constructive Proofs}
This is another kind of an existence proof. In this method, we do not
explicitly find a value of 'x' such that $P(x)$ is true. Instead, we show that
there must exist some x for which $P(x)$ is true. This might seem like a
confusing definition, so let us use this in an example:

\begin{question}
    Show that there exists a pair of irrational numbers a and b such that $c =
    a^{b}$ is rational.
\end{question}

\begin{remark}
    This is a very famous and interesting problem in the field of proofs and
    discrete mathematics. Let us look at the solution:
\end{remark}

\begin{proof}
    We know that $\sqrt 2$ is irrational. Let us define $c = (\sqrt 2)^{\sqrt
    2}$.

    Additionally, let us define $d = c ^{\sqrt2}$.

    Then, $d = (\sqrt 2) ^{\sqrt 2 \times \sqrt 2} = (\sqrt 2) ^{2} = 2$ which
    is rational.
\end{proof}

Note that we did not explicitly find the values of a and b in the above
example. We simply showed that there could exist a pair of irrational numbers
in which $a^{b}$ could be rational.

\textbf{\emph{Trivia: }} Did you know that there is a constructive
proof for the above mentioned example? It is quite hard to find the right
examples for a and b, but it is possible. I haven't included the answers, but
do think about it!
