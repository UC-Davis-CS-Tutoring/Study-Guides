\documentclass[]{book}
\usepackage{lmodern}
\usepackage{amssymb,amsmath}
\usepackage{ifxetex,ifluatex}
\usepackage{fixltx2e} % provides \textsubscript
\ifnum 0\ifxetex 1\fi\ifluatex 1\fi=0 % if pdftex
  \usepackage[T1]{fontenc}
  \usepackage[utf8]{inputenc}
\else % if luatex or xelatex
  \ifxetex
    \usepackage{mathspec}
  \else
    \usepackage{fontspec}
  \fi
  \defaultfontfeatures{Ligatures=TeX,Scale=MatchLowercase}
\fi
% use upquote if available, for straight quotes in verbatim environments
\IfFileExists{upquote.sty}{\usepackage{upquote}}{}
% use microtype if available
\IfFileExists{microtype.sty}{%
\usepackage{microtype}
\UseMicrotypeSet[protrusion]{basicmath} % disable protrusion for tt fonts
}{}
\usepackage[margin=1in]{geometry}
\usepackage{hyperref}
\PassOptionsToPackage{usenames,dvipsnames}{color} % color is loaded by hyperref
\hypersetup{unicode=true,
            pdftitle={ECS36C: Data Structures},
            pdfauthor={Aakash Prabhu, Aaron Kaloti, Erica Nguyen},
            colorlinks=true,
            linkcolor=Maroon,
            citecolor=Blue,
            urlcolor=Blue,
            breaklinks=true}
\urlstyle{same}  % don't use monospace font for urls
\usepackage{natbib}
\bibliographystyle{apalike}
\usepackage{longtable,booktabs}
\usepackage{graphicx,grffile}
\makeatletter
\def\maxwidth{\ifdim\Gin@nat@width>\linewidth\linewidth\else\Gin@nat@width\fi}
\def\maxheight{\ifdim\Gin@nat@height>\textheight\textheight\else\Gin@nat@height\fi}
\makeatother
% Scale images if necessary, so that they will not overflow the page
% margins by default, and it is still possible to overwrite the defaults
% using explicit options in \includegraphics[width, height, ...]{}
\setkeys{Gin}{width=\maxwidth,height=\maxheight,keepaspectratio}
\IfFileExists{parskip.sty}{%
\usepackage{parskip}
}{% else
\setlength{\parindent}{0pt}
\setlength{\parskip}{6pt plus 2pt minus 1pt}
}
\setlength{\emergencystretch}{3em}  % prevent overfull lines
\providecommand{\tightlist}{%
  \setlength{\itemsep}{0pt}\setlength{\parskip}{0pt}}
\setcounter{secnumdepth}{5}
% Redefines (sub)paragraphs to behave more like sections
\ifx\paragraph\undefined\else
\let\oldparagraph\paragraph
\renewcommand{\paragraph}[1]{\oldparagraph{#1}\mbox{}}
\fi
\ifx\subparagraph\undefined\else
\let\oldsubparagraph\subparagraph
\renewcommand{\subparagraph}[1]{\oldsubparagraph{#1}\mbox{}}
\fi

%%% Use protect on footnotes to avoid problems with footnotes in titles
\let\rmarkdownfootnote\footnote%
\def\footnote{\protect\rmarkdownfootnote}

%%% Change title format to be more compact
\usepackage{titling}

% Create subtitle command for use in maketitle
\newcommand{\subtitle}[1]{
  \posttitle{
    \begin{center}\large#1\end{center}
    }
}

\setlength{\droptitle}{-2em}

  \title{ECS36C: Data Structures}
    \pretitle{\vspace{\droptitle}\centering\huge}
  \posttitle{\par}
  \subtitle{\emph{Study Guide}}
  \author{Aakash Prabhu, Aaron Kaloti, Erica Nguyen}
    \preauthor{\centering\large\emph}
  \postauthor{\par}
      \predate{\centering\large\emph}
  \postdate{\par}
    \date{Version 1.0}

\usepackage{booktabs}
\usepackage{amsthm}
\makeatletter
\def\thm@space@setup{%
  \thm@preskip=8pt plus 2pt minus 4pt
  \thm@postskip=\thm@preskip
}
\makeatother

\begin{document}
\maketitle

{
\hypersetup{linkcolor=black}
\setcounter{tocdepth}{1}
\tableofcontents
}
\hypertarget{preface}{%
\chapter*{Preface}\label{preface}}
\addcontentsline{toc}{chapter}{Preface}

This study guide is written with the intention to help students review
topics easily and quickly before an exam, or a quiz. If you are enrolled
in ECS 36C this quarter, please don't use this study guide as your
primary means of preparation for an exam. This is only meant to review
topics and this will \textbf{not} replace a textbook or lecture notes
provided by your instructor.

Data Structures and Algorithms is probably one of the most important
courses that one would take during their undergraduate life. Because
these concepts are used everywhere, they always show up on technical
interviews! Since most students will interview for either internship or
full-time positions, we will try and indicate what topics we got tested
on during interviews.

This study guide goes over each and every topic, and possibly more than
what you might see in this course. The topics presented in this guide
are well within the scope of this course. We have broken down this study
guide into three parts, for ease of understanding: \emph{Data
Structures}, \emph{Graph Algorithms}, and \emph{Sorting Algorithms}.
Each of these parts are broken down into simpler topics and their use
cases are explained in simple terms. Since this is a data structures and
algorithms course, you are expected to know the time complexities of the
different data structures you study. We have provided these run times
along with a detailed explanation/proof for each Data Structure and/or
Algorithm you will cover in this class.

This study guide largely makes use of examples and uses examples to
illustrate the working of the different Data Structures and Algorithms
you will see. Wherever neccessary, we have explained certain concepts
using code or pseudocode. We will primarily use \emph{C++} to illustrate
such implementations of algorithms.

The authors of this study guide have all previously taken this course
and did exceedingly well. Additionally, we have all tutored this class
and held review sessions of our own for over 2 years now. We have
tutored and helped countless students in this grueling course. We know
those trivial tips and tricks that might just help you do better on an
exam!

If you spot a typo or a conceptual error, please contact us at
\texttt{ucdcstutoring\ {[}at{]}\ gmail\ \{dot\}\ com}. Since this is the
first time we are writing a study guide for this course, your
suggestions and feedback will be greatly appreciated.

Before we sign this preface and let you get reading with the real
material, here's a small piece of information that you should take from
this class: \emph{Never ever use a B-Tree in RAM}. If you don't know
what that means, you're about to find out soon!

Good Luck,

\emph{The CS Tutoring Committee}

\hypertarget{introduction}{%
\chapter{Introduction}\label{introduction}}

TODO: Write Intro

\bibliography{book.bib,packages.bib}


\end{document}
